% *created  "Fri Jun  5 12:11:18 2020" *by "Paul E. Black"
% *modified "Wed Dec 14 10:59:58 2022" *by "Paul E. Black"

\documentclass[12pt]{article}
\usepackage{amsmath}
\usepackage{amsfonts}   % if you want the fonts
\usepackage{amssymb}    % if you want extra symbols
\usepackage{graphicx}   % need for figures
\usepackage{xcolor}
\usepackage{bm}
\usepackage{secdot}		
\usepackage{mathptmx}
\usepackage{float}
\usepackage[utf8]{inputenc}
\usepackage{textcomp}
\usepackage[hang,flushmargin,bottom]{footmisc} % footnote format

\usepackage{titlesec}
\titleformat{\section}{\normalsize\bfseries}{\thesection.}{1em}{}	% required for heading numbering style
\titleformat*{\subsection}{\normalsize\bfseries}

\usepackage{tocloft}	% change typeset, titles, and format list of appendices/figures/tables
\renewcommand{\cftdot}{}	
\renewcommand{\contentsname}{Table of Contents}
\renewcommand{\cftpartleader}{\cftdotfill{\cftdotsep}} % for parts
\renewcommand{\cftsecleader}{\cftdotfill{\cftdotsep}}
\renewcommand\cftbeforesecskip{\setlength{4pt}{}}
\addtolength{\cftfignumwidth}{1em}
\renewcommand{\cftfigpresnum}{\figurename\ }
\addtolength{\cfttabnumwidth}{1em}
\renewcommand{\cfttabpresnum}{\tablename\ }
\setlength{\cfttabindent}{0in}    %% adjust as you like
\setlength{\cftfigindent}{0in} 

\usepackage{enumitem}         % to control spacing between bullets/numbered lists

\usepackage[numbers,sort&compress]{natbib} % format bibliography 
\renewcommand{\bibsection}{}
\setlength{\bibsep}{0.0pt}

% hyphenate URLs in more places (must precede hyperref)
\usepackage[hyphens,obeyspaces]{url}
\usepackage[hidelinks]{hyperref}
\hypersetup{
	colorlinks = true,
urlcolor ={blue},
citecolor = {.},
linkcolor = {.},
anchorcolor = {.},
filecolor = {.},
menucolor = {.},
runcolor = {.}
pdftitle={},
pdfsubject={},
pdfauthor={},
pdfkeywords={}
}
\urlstyle{same}

\usepackage{epstopdf} % converting EPS figure files to PDF

\usepackage{fancyhdr, lastpage}	% formatting document, calculating number of pages, formatting headers
\setlength{\topmargin}{-0.5in}
\setlength{\headheight}{39pt}
\setlength{\oddsidemargin}{0.25in}
\setlength{\evensidemargin}{0.25in}
\setlength{\textwidth}{6.0in}
\setlength{\textheight}{8.5in}

\usepackage{caption} % required for Figure labels
\captionsetup{font=small,labelfont=bf,figurename=Fig.,labelsep=period,justification=raggedright} 

% put a light gray ``DRAFT'' diagonally across each page
\usepackage{draftwatermark}
\usepackage{datetime} % for \shortmonthname
\SetWatermarkText{DRAFT \number\day\ \shortmonthname}
% bigger percentage is closer to white
\SetWatermarkColor[gray]{0.9}
\SetWatermarkFontSize{3cm}
\SetWatermarkAngle{55}
\SetWatermarkHorCenter{11cm}

\usepackage{xcolor} % for colored text and backgrounds

%%%%%%%%%%% !!!!!! REQUIRED - FILL OUT METADATA FOR NISTIR HERE !!!!!!!! %%%%%%%%%%%%%%
%  	Report Number - fill in Report Number sent to you (see info below)
%   DOI Statement - fill in DOI sent to you 
%   Month Year - fill in Month and Year of Publication
%%%%%%%%%%%%%%%%%%%%%%%%%%%%%%%%%%%%%%%%%%%%%%%%%%%%%%%%%%%%%%%%%%%%%%%%%%%%%%%%%%%%%%
\newcommand{\pubnumber}{XXXX}
\newcommand{\DOI}{https://doi.org/10.6028/NIST.IR.XXXX}
\newcommand{\monthyear}{October 2020}
\newcommand{\paperTitle}{Vulnerability Test Suite Generator (VTSG) Version 3}

%%%%%%%%%%%%%%%%%%%%%%%%%%%%%%%%%%%%%%%%%%%%%%%%%%%%%%%%%%%%%%%%%%%%
%   	BEGIN DOCUMENT 
%%%%%%%%%%%%%%%%%%%%%%%%%%%%%%%%%%%%%%%%%%%%%%%%%%%%%%%%%%%%%%%%%%%%

% correct bad hyphenation here
\hyphenation{e-qua-tion}

% list of commands for horizontal spacing---and there are a LOT of them---and
% examples of each one
% https://tex.stackexchange.com/questions/74353/what-commands-are-there-for-horizontal-spacing

% to wrap paragraph in table cells
\usepackage{makecell}

% typeset C#
% According to the ECMA-334 C# Language Specification, 5th
% Edition, Sec. 5, Acronyms and abbreviations, ``The name 
% C# is written as the LATIN CAPITAL LETTER C (U+0043) followed 
% by the NUMBER SIGN # (U+0023).'' \cite{ECMA-CSharp2017}
% https://www.ecma-international.org/publications/standards/Ecma-334.htm
% Dec 2017, Accessed: 12 June 2020
\newcommand{\CSharp}{C{\fontseries{b}\selectfont\#}}
% another possibility is C\nolinebreak\#\

% zero-width space, so "--" does not become en dash
\newcommand{\zws}{\hspace{0pt}}

\begin{document}
	\urlstyle{rm} % Format style of \url   

%%%%%%%%%%%%%%%%%%%%%%%%%%%%%%%%%%%%%%%%%%%%%%%%%%%%%%%%%%%%%%%%%%%%
%   Cover Page is REQUIRED and must contain the information 
%	displayed here, at a minimum. Additional artwork may be included 
%	(e.g., official project/conference logo, etc.).
%	Pub Number automated based on metadata
%%%%%%%%%%%%%%%%%%%%%%%%%%%%%%%%%%%%%%%%%%%%%%%%%%%%%%%%%%%%%%%%%%%%
\begin{titlepage}
\begin{flushright}
%%%%%%%%%%%%%%%%%%%%%%%%%%%%%%%%%%%%%%%%%%%%%%%%%%%%%%%%%%%%%%%%%%%%
% 	Automated based on metadata - delete if not applicable
%%%%%%%%%%%%%%%%%%%%%%%%%%%%%%%%%%%%%%%%%%%%%%%%%%%%%%%%%%%%%%%%%%%%
\LARGE{\textbf{NISTIR \pubnumber}}\\
\vfill
%%%%%%%%%%%%%%%%%%%%%%%%%%%%%%%%%%%%%%%%%%%%%%%%%%%%%%%%%%%%%%%%%%%%
%	Title 
%%%%%%%%%%%%%%%%%%%%%%%%%%%%%%%%%%%%%%%%%%%%%%%%%%%%%%%%%%%%%%%%%%%%
\Huge{\textbf{\paperTitle}}\\
    \vfill
%%%%%%%%%%%%%%%%%%%%%%%%%%%%%%%%%%%%%%%%%%%%%%%%%%%%%%%%%%%%%%%%%%%%
%	Authors - add complete list of authors, affiliations will be 
%   added on title page
%%%%%%%%%%%%%%%%%%%%%%%%%%%%%%%%%%%%%%%%%%%%%%%%%%%%%%%%%%%%%%%%%%%%
    \large Paul E. Black \\
    \large William Mentzer\\
    \large Elizabeth Fong \\
    \large Bertrand Stivalet
\vfill
%%%%%%%%%%%%%%%%%%%%%%%%%%%%%%%%%%%%%%%%%%%%%%%%%%%%%%%%%%%%%%%%%%%%
%	The DOI is automated based on metadata.	
%%%%%%%%%%%%%%%%%%%%%%%%%%%%%%%%%%%%%%%%%%%%%%%%%%%%%%%%%%%%%%%%%%%%
\normalsize This publication is available free of charge from:\\
\DOI\\
\vfill
%%%%%%%%%%%%%%%%%%%%%%%%%%%%%%%%%%%%%%%%%%%%%%%%%%%%%%%%%%%%%%%%%%%%
%	NIST LOGO - keep as-is
%%%%%%%%%%%%%%%%%%%%%%%%%%%%%%%%%%%%%%%%%%%%%%%%%%%%%%%%%%%%%%%%%%%%

\includegraphics[width=0.3\linewidth]{NIST-logo.eps}\\ 
 
  
\end{flushright}
\end{titlepage}
\begin{titlepage}
%%%%%%%%%%%%%%%%%%%%%%%%%%%%%%%%%%%%%%%%%%%%%%%%%%%%%%%%%%%%%%%%%%%%
%	Title Page is REQUIRED
%%%%%%%%%%%%%%%%%%%%%%%%%%%%%%%%%%%%%%%%%%%%%%%%%%%%%%%%%%%%%%%%%%%%
\begin{flushright}
%%%%%%%%%%%%%%%%%%%%%%%%%%%%%%%%%%%%%%%%%%%%%%%%%%%%%%%%%%%%%%%%%%%%
%   Publication Series & Number - automated
%%%%%%%%%%%%%%%%%%%%%%%%%%%%%%%%%%%%%%%%%%%%%%%%%%%%%%%%%%%%%%%%%%%%
\LARGE{\textbf{NISTIR \pubnumber}}\\
\vfill 
%%%%%%%%%%%%%%%%%%%%%%%%%%%%%%%%%%%%%%%%%%%%%%%%%%%%%%%%%%%%%%%%%%%%
%	Title 
%%%%%%%%%%%%%%%%%%%%%%%%%%%%%%%%%%%%%%%%%%%%%%%%%%%%%%%%%%%%%%%%%%%%
\Huge{\textbf{\paperTitle}}\\
    \vfill
%%%%%%%%%%%%%%%%%%%%%%%%%%%%%%%%%%%%%%%%%%%%%%%%%%%%%%%%%%%%%%%%%%%%
%	Author Order and Grouping. Always identify the primary author/creator first
% (s/he does not have to be a NIST author). For publications with multiple authors,
% group authors by their organizational affiliation. The organizational groupings and
% the names within each grouping should generally be ordered by decreasing level of
% contribution.
%	For non-NIST authors, list their city and state below their organization name.
%	For NIST authors, include the Division and Laboratory names (but do not
% include their city and state).
%%%%%%%%%%%%%%%%%%%%%%%%%%%%%%%%%%%%%%%%%%%%%%%%%%%%%%%%%%%%%%%%%%%%
    \normalsize 
    Paul E. Black\\
     \textit{Software and Systems Division}\\
     \textit{Information Technology Laboratory}\\
     \vspace{12pt}
    William Mentzer\\
     \textit{California State University}\\
     \textit{San Bernardino, California}\\
     \vspace{12pt}
    Elizabeth Fong \\
     \textit{affiliation}\\
     \textit{location}\\
     \vspace{12pt}
    Bertrand Stivalet \\
     \textit{affiliation}\\
     \textit{location}
\vfill
%%%%%%%%%%%%%%%%%%%%%%%%%%%%%%%%%%%%%%%%%%%%%%%%%%%%%%%%%%%%%%%%%%%%
%   DOI Statement - automated
%%%%%%%%%%%%%%%%%%%%%%%%%%%%%%%%%%%%%%%%%%%%%%%%%%%%%%%%%%%%%%%%%%%%
\normalsize This publication is available free of charge from:\\
\DOI\\
\vfill
%%%%%%%%%%%%%%%%%%%%%%%%%%%%%%%%%%%%%%%%%%%%%%%%%%%%%%%%%%%%%%%%%%%%
%   Date - Month and Year - automated
%%%%%%%%%%%%%%%%%%%%%%%%%%%%%%%%%%%%%%%%%%%%%%%%%%%%%%%%%%%%%%%%%%%%
\normalsize \monthyear
\vfill
%%%%%%%%%%%%%%%%%%%%%%%%%%%%%%%%%%%%%%%%%%%%%%%%%%%%%%%%%%%%%%%%%%%%
%  Department of Commerce LOGO - leave as-is
%%%%%%%%%%%%%%%%%%%%%%%%%%%%%%%%%%%%%%%%%%%%%%%%%%%%%%%%%%%%%%%%%%%%	

\includegraphics[width=0.18\linewidth]{DoC-logo.eps}\\ 
 \vfill
%%%%%%%%%%%%%%%%%%%%%%%%%%%%%%%%%%%%%%%%%%%%%%%%%%%%%%%%%%%%%%%%%%%%
%  Department of Commerce & NIST Leadership 
%	will be updated as changes occur
%%%%%%%%%%%%%%%%%%%%%%%%%%%%%%%%%%%%%%%%%%%%%%%%%%%%%%%%%%%%%%%%%%%%
\footnotesize U.S. Department of Commerce\\ 
\textit{Wilbur L. Ross, Jr., Secretary}\\
\vspace{10pt}
National Institute of Standards and Technology\\ 
\textit{Walter Copan, NIST Director and Undersecretary of Commerce for Standards and Technology}  
\end{flushright}
\end{titlepage}
\begin{titlepage}
%%%%%%%%%%%%%%%%%%%%%%%%%%%%%%%%%%%%%%%%%%%%%%%%%%%%%%%%%%%%%%%%%%%%
%   Disclaimer/CODEN page - required
%%%%%%%%%%%%%%%%%%%%%%%%%%%%%%%%%%%%%%%%%%%%%%%%%%%%%%%%%%%%%%%%%%%%
\begin{flushright}
\footnotesize  Certain commercial entities, equipment, or materials may be identified
in this document in order to describe an experimental procedure or concept
adequately. Such identification is not intended to imply recommendation or
endorsement by the National Institute of Standards and Technology, nor is it intended
to imply that the entities, materials, or equipment are necessarily the best
available for the purpose.\\

\vfill
%%%%%%%%%%%%%%%%%%%%%%%%%%%%%%%%%%%%%%%%%%%%%%%%%%%%%%%%%%%%%%%%%%%%
%   This section automated - do not change
%%%%%%%%%%%%%%%%%%%%%%%%%%%%%%%%%%%%%%%%%%%%%%%%%%%%%%%%%%%%%%%%%%%%
\normalsize \textbf{National Institute of Standards and Technology \\ Internal Report \pubnumber\\ 
Natl. Inst. Stand. Technol. Intern. Rep. \pubnumber, \pageref{LastPage} pages (\monthyear)} \\
\vspace{12pt}
\textbf{This publication is available free of charge from: \DOI}
\vfill
\end{flushright}
\end{titlepage}
%%%%%%%%%%%%%%%%%%%%%%%%%%%%%%%%%%%%%%%%%%%%%%%%%%%%%%%%%%%%%%%%%%%%
%   Start front matter - page number starts with "i"
%%%%%%%%%%%%%%%%%%%%%%%%%%%%%%%%%%%%%%%%%%%%%%%%%%%%%%%%%%%%%%%%%%%%
\pagenumbering{roman}

\section*{Abstract}
\normalsize
The Vulnerability Test Suite Generator (VTSG) can create vast numbers of
synthetic programs with and without specific flaws or vulnerabilities.
It was designed by the Software Assurance Metrics and Tool Evaluation (SAMATE) team
and originally implemented by students from TELECOM Nancy.
The latest version is structured to be able to generate vulnerable and nonvulnerable
synthetic programs expressing specific flaws in \emph{any} programming language.
It has libraries to generate PHP, \CSharp, and Python programs.
This document may help if you are trying to generate test cases written in 
PHP, \CSharp, or Python, adding new complexities or flaws or vulnerability, or
modifying VTSG Version 3 to generate test
cases in other programming languages.

\section*{Key words}
\normalsize Software assurance; static analyzer; test case generator; 
software vulnerabilities.

% push next part to the bottom of the page
\vfill

This document was written at the National Institute of Standards and
Technology by employees of the Federal Government in the course of
their official duties.  Pursuant to title 17 Section 105 of the United
States Code this is not subject to copyright protection and
is in the public domain.


We would appreciate acknowledgment if this document is used.

\pagebreak
%%%%%%%%%%%%%%%%%%%%%%%%%%%%%%%%%%%%%%%%%%%%%%%%%%%%%%%%%%%%%%%%%%%%
%   Table of Contents is required
% 	List of Tables & Figures required if more than 5 tables/figures
%%%%%%%%%%%%%%%%%%%%%%%%%%%%%%%%%%%%%%%%%%%%%%%%%%%%%%%%%%%%%%%%%%%%
\begin{center}
\tableofcontents
\listoftables
\listoffigures
\end{center}
\pagebreak
%%%%%%%%%%%%%%%%%%%%%%%%%%%%%%%%%%%%%%%%%%%%%%%%%%%%%%%%%%%%%%%%%%%%
%   Start body of text - page number starts with "1"
%%%%%%%%%%%%%%%%%%%%%%%%%%%%%%%%%%%%%%%%%%%%%%%%%%%%%%%%%%%%%%%%%%%%

\newpage

\section{Introduction}
\pagenumbering{arabic}

The Vulnerability Test Suite Generator (VTSG) generates collections 
of vulnerable and non-vulnerable
synthetic programs expressing specific flaws.  
The programs can be used as 
test cases to evaluate static analyzers. 
Each test case targets one flaw. There are two types of test cases:
cases with flawed code (unsafe), leading to a vulnerability, and cases that
have similar behavior, but have the flaw corrected, that is, no
vulnerability (safe).  Exactly corresponding vulnerable and non-vulnerable
cases could, in theory, be generated in pairs. However, since each test case
is generated separately, there is no exact correspondence between 
cases.

NIST's Software Assurance Reference Dataset (SARD) has suites of paired test cases in
the Juliet test suites for C/C++ and Java, which are available as SARD test suites
108 and 109.  These and many other test suites are available at
\href{https://samate.nist.gov/SARD/testsuite.php}
     {https://samate.nist.gov/ SARD/testsuite.php}.

The generator is written in Python 3.
The VTSG git repository is at
\href{https://github.com/usnistgov/VTSG}{https://github.com/ usnistgov/VTSG}.
The list of files and the README.md file are given in App.~\ref{gitContent}.


\subsection{History} 

VTS version 1 only generated \CSharp\ programs. VTS version 2 generated PHP
programs~\cite{StivaletFongVTSPHP2016} in addition.  Version 2 is more
customizable to generate other programming languages.
VTSG version 3 (V3) systematically maintains indentation, so also
generates Python programs.  VTSG V3 produces manifest files in the
Static Analysis Results Interchange Format (SARIF) 
Version 2.1.0~\cite{SARIF2.1.0} format.

Readers can download the PHP and \CSharp\ test cases generated 
by earlier versions as SARD test suites 103 and 105 from
\href{https://samate.nist.gov/SARD/testsuite.php}{https://samate.nist.gov/SARD/testsuite.php}.


\subsection{Install Supporting Packages}

\noindent The following instructions are provided for users who may not have these
packages already installed on their Linux machines. Users who already have these
packages may skip this section.

\noindent To download files from Github, first install the \emph{Git} package.
Here is the command to install it:

\begin{verbatim}
    sudo apt-get install git    
\end{verbatim}

\noindent VTSG is written in Python 3, so Python 3 must be installed, too.
Here is the command to install it:

\begin{verbatim}
    sudo apt-get install python3
\end{verbatim}

\noindent To download Python source code packages, install the \emph{pip
Python} package manager.
Here is the command to install it:

\begin{verbatim}
    sudo apt-get install python-pip
\end{verbatim}

\noindent One may also have to install the \emph{pip Python 3} package manager.
Here is the command to install it:

\begin{verbatim}
    sudo apt-get install python3-pip
\end{verbatim}

\noindent Another way to install \emph{pip} is:

\begin{verbatim}
    sudo python3 -m pip install --upgrade pip
\end{verbatim}

\noindent To validate \CSharp\ test cases, install \emph{mono} and \emph{mcs}
to run them. The Mono project created \emph{mono} as an open source
platform, which
implements the .NET Framework. Class libraries and \CSharp\ compilation are enabled
by \emph{mcs} (\href{http://github.com/mono/mono}{http://github.com/mono/mono}).

\noindent Here is the command to install it:

\begin{verbatim}
    sudo apt-get install mono-complete
\end{verbatim}

\noindent Here is the command to install it:

\begin{verbatim}
    sudo apt-get install mcs
\end{verbatim}

\subsection{Install VTSG}

\noindent To copy the generic VTSG from GitHub to a local Linux machine, change to a
directory under which you want to install VTSG.

\noindent Looking at the GitHub website, one will see a green box labeled ``Code''.
See Fig.~\ref{fig:clone button}.
Click on it, then click on the ``copy'' icon to copy the web URL.

\begin{figure}[htbp]
  \includegraphics[width=1\linewidth]{fig_clone_tab.png}
  \caption{usnistgov/VTSG: button to clone repository.}
  \label{fig:clone button}
\end{figure}

\noindent Here is the command to copy the source code and other material to the local
directory:

\begin{verbatim}
    git clone https://github.com/usnistgov/VTSG.git
\end{verbatim}

\noindent If one types the \verb|ls| command, one will see that the \textbf{VTSG}
directory was created.

\noindent Go into that directory, using this command:

\begin{verbatim}
    cd VTSG
\end{verbatim}

\noindent To install the dependencies, use this command:

\begin{verbatim}
    pip3 install --user -r requirements.txt
\end{verbatim}


\subsection{Users}

There are two groups who will typically use VTSG. The first group comprises people
requiring test cases written in PHP, \CSharp, or Python to evaluate a static
analyzer. These users must know how to invoke VTSG with the command line interface
and retrieve the appropriate sample from the generated and categorized folders.

The second targeted audience of the VTSG is comprised of people wishing to generate
test cases using a programming language other than the languages currently
supported.  The second group of users must create new templates for the program
using the XML tags, execute VTSG with the command line interface,
and retrieve the samples from the generated and categorized file folders.

\subsection{Vulnerabilities Currently Encoded in VTSG Files}

Vulnerabilities are encoded in the language files, see Sec.~\ref{sec:source files}.
Some of the OWASP Top 10~\cite{OWASPTop10-2017} and
Common Weakness Enumerations (CWEs)~\cite{CWE} are encoded.
See App.~\ref{sec:CSharp language} and \ref{sec:PHP language} for details.


\section{Command Line Interface}
\label{sec:command line interface}

For users who wish to generate PHP, \CSharp, or Python test suites, a command line
interface
can generate all test cases or a specific group of test cases based on several
options.  For example, the user can generate vulnerable or non-vulnerable test cases
based on selected flaws or groups of flaws, for example, OWASP categories.
The user must specify the
programming language.  The invocation command looks like this:
\begin{verbatim}
$ python3 vtsg.py -l {php,cs,py} <options>
\end{verbatim}

% Overleaf complains that textcomp doesn't provide \textlangle
% and \textrangle in the font family.  So we define our own.
\newcommand{\texlangle}{$\langle$}
\newcommand{\texrangle}{$\rangle$}
Where \verb|<options>| can be selected from
Table~\ref{tab:command line options}.

\begin{table}[H]
\centering
\begin{tabular}{|l|l|}
\hline
-h -\zws-help & Show help and quit \\
\hline
-\zws-version & Show version number and quit \\
\hline
-l LANGUAGE 
-\zws-language=LANGUAGE &
\makecell[l]{Language of generated cases. \\
Currently one of php, for PHP \\
cases, cs, for \CSharp\ cases, or py, \\
for Python cases. 
See Sec.~\ref{sec: directory structure}.} \\
\hline
\makecell[l]{-g GROUP[,GROUP]* \\
-\zws-group=GROUP[,GROUP]*} &
\makecell[l]{Only generate cases with \\
vulnerabilities in the specified \\
group(s). See Sec.~\ref{sec:sink modules}.} \\
\hline
-f Flaw[,Flaw]*
-\zws-flaw=Flaw[,Flaw]* &
\makecell[l]{Only generate cases with \\
the specified flaw(s). \\
See Sec.~\ref{sec:sink modules}.} \\
\hline
-s
-\zws-safe &
Only generate non-vulnerable cases \\
\hline
-u
-\zws-unsafe &
Only generate vulnerable cases \\
\hline
-r DEPTH
-\zws-depth=DEPTH &
\makecell[l]{Maximum nested depth of \\
complexities (Default: 1) See \\ 
Sec.~\ref{sec:depth of complexities}.} \\
\hline
-n NUMBER
-\zws-number-generated=NUMBER &
\makecell[l]{Maximum number of sink, filter, \\
input, and exec query combina- \\
tions to generate. (Default: -1, \\
meaning all) See below for \\
explanation.} \\

\hline
\makecell[l]{-t TEMPLATE\_DIRECTORY \\
-\zws-template-directory=TEMPLATE\_DIRECTORY} &
\makecell[l]{The language templates directory. \\
  (Default: src/templates)} \\

\hline
-d
-\zws-debug &
\makecell[l]{for programmer use} \\
\hline
\end{tabular}
\caption{Options for Command Line Invocation}
\label{tab:command line options}
\end{table} 

\subsection{Explanation of Options}

The default is to generate both the unsafe (buggy or vulnerable)
test cases and the safe (not buggy) test cases.  You can select
either only safe (-s) or only unsafe (-u) cases.  The options are
mutually exclusive.

The -n (number-generated) option has limited utility.  When the
specified number of sink, filter, input (and exec query, if needed)
combinations are generated, VTSG terminates.  The default, -1,
means generate all combinations.

Each combination of sink, filter, input, and exec query is elaborated
with DEPTH nested complexities.  Suppose there are 5 complexities
and VTSG is invoked with \verb|-r 2|.  Each combination will yield
1 (no complexities) + 5 (each complexity, not nested) + 
25 (each complexities nested within every complexity) = 31 test cases.  
Hence VTSG will generate \emph{far} more test cases than the number
given with the -n option.

\subsection{Example Invocations}

Show the help message:
\begin{verbatim}
$ python3 vtsg.py --help
\end{verbatim}

Generate all PHP test cases:
\begin{verbatim}
$ python3 vtsg.py -l php
\end{verbatim}

This takes about 25 minutes.  (Generating all the \CSharp\ cases takes about four
minutes.)

Generate a \CSharp\ (\verb|-l cs|) test suite made of vulnerable (unsafe) test
cases (\verb|-u|) with SQL injection vulnerabilities (\verb|--flaw=CWE_89|)
and up to 3 nested levels of complexity (\verb|-r 3|).
\begin{verbatim}
$ python3 vtsg.py -l cs -r 3 --flaw=CWE_89 -u
\end{verbatim}
 
\section{Overview of VTSG}

\begin{figure}[htbp]
  \includegraphics[width=1\linewidth]{fig_VTSG_overview.png}
  \caption{Overview of VTSG test case generation process. The Template
  specifies how pieces are assembled. Input, Filter, Complexities,
  Sink, and ExecQuery modules provide alternative code.  
  An example of
  a generated test case is in 
  Figs.~\ref{fig:example main file} and 
  \ref{fig:example aux file}.}
  \label{fig:VTS operation overview}
\end{figure}

Please note that this document describes two different program structures: the
structure of VTSG itself and the structure of test cases that it generates.

\subsection{Overview of Test Case Generation}

VTSG V3 generates test cases from information in Template, Input,
Filter, Sink, Complexity and ExecQuery files 
(see Fig.~\ref{fig:VTS operation overview}):
\begin{itemize}
 \item Template is the overall structure of each program.
 \item Input is the source of untrusted data in the program, e.g., 
 command line, variable, files, or form methods.
 \item Filter filters the input with functions or methods such as sanitization
   functions, casting, or deprecated functions.
 \item Sink is where a sensitive operation, such as a database query, 
 is executed with potentially untrusted input and where the 
 vulnerability is triggered.
 \item ExecQuery is an additional piece of code that is mandatory to 
 trigger the vulnerability.
 \item Complexity are additional data flow or control flow complications that are
   worked into the structure of the program to exercise tools' abilities.
\end{itemize}
The content of these files is detailed in
Sec.~\ref{sec:source files}.
Details of the generation process are explained in
Sec.~\ref{sec: generation detail}.

\label{sec:case directory structure}
When invoked to generate test cases, VTSG creates a directory with all the results.
The directory is named for the date and time created, for example,
\verb|TestSuite_03-08-2022_| \verb|16h46m35|.
The language directory, \verb|PHP|,
\verb|Csharp|, or \verb|Python| is created in this directory.
This language name comes from the \verb|name|
in the \verb|file_template.xml| file.  See Sec.~\ref{sec: file template}.

Under the language directory, VTSG creates one directory for each flaw group, for
instance, \verb|OWASP_a1| or \verb|OWASP_a4|.  These come from the \verb|flaw_group|
in the \verb|sinks.xml| file; see Sec.~\ref{sec:sink modules}.
Under each flaw group directory is a subdirectory for each specific flaw, for
instance, \verb|CWE_78| or \verb|CWE_89|.  These come from the \verb|flaw_type|
entries, which are also in the \verb|sinks.xml| file.
VTSG also creates a manifest file of all the test cases generated for each flaw
group, named
\verb|manifest.xml|.

If the \verb|flaw_group| is missing or empty, subdirectories for flaw types are
created immediately under the language directory.

Under the flaw directory are directories for safe or unsafe
(vulnerable) test cases, depending on
which are generated.

\subsection{VTSG Directory Structure}
\label{sec: directory structure}

All the language information files are accessed from a single subdirectory.  The
default is \verb|src/templates|.  Another directory may be given with the -t command
line option, see Table~\ref{tab:command line options}.

Under this is one subdirectory for each language.  VTSG chooses the
subdirectory based on the language given in the \verb|-l| command line
option.  Each language subdirectory has six files for that language.
The files are \verb|file_template.xml|, \verb|inputs.xml|,
\verb|complexities.xml|, \verb|filters.xml|, \verb|sinks.xml|,
and \verb|exec_queries.xml|.
The files are described in Sec.~\ref{sec:source files}.
To add another language,
simply create a subdirectory for that language with the six description
files.

The \verb|src/templates| directory has \verb|file_rights.txt|, which is
copied into each generated test case to declare license rights and
authorship, see Sec.~\ref{sec: file template}, and a 
\verb|dtd| subdirectory, which has document type definitions (DTDs)
for all of the XML files.


\subsection{Details of Test Case Generation}
\label{sec: generation detail}

Each test case is constructed based on the file template, as shown
in Fig.~\ref{fig:VTS operation overview}. Test cases are 
programs in a specific program language.  
Each test case is generated by
assembling the modules according to the Template.  The
template may direct construction of a simple test case with just
an Input and a Sink.
The Filter code may be embedded in data and control flow
Complexity code.

VTSG generates test cases with two broad steps.  First, VTSG selects
sink, filter, input, exec query, and complexities that are
compatible with each other and consistent with any flaw group or
flaw constraints the user gives on the command line.  Second, VTSG
composes test case source code from the selected modules, 
synthesizing variable and functions names, and writes the file(s).

The code structure is roughly
\begin{verbatim}
    for each specified sink
        for each filter
            for each input
                for each exec query
                    for up to DEPTH combinations of each complexity
                        compose a test case with these modules
\end{verbatim}
The code is more complicated because only compatible modules are 
selected.
In addition, some sinks do not need any input or filtering at all,
see Sec.~\ref{sec:sink modules}.
The
code is structured as a series of function calls to allow types
of modules to be skipped.  Here is a slightly more detailed
overview of those steps, which are in \verb|generator.py|:
\begin{verbatim}
    for each sink:
        if this sink is the type specified:
            use this sink
            if input is needed:
                select_filtering()
            else:
                select_exec_queries()
    
    def select_filtering():
        for each filter:
            if filter is compatible with sink:
                use this filter
                select_input()
    
    def select_input()
        for each input:
            if input is compatible with filter and sink:
                use this input
                select_exec_queries()

    def select_exec_queries()
        if sink needs exec_query:
            for each exec_query:
                if exec_query is compatible with sink:
                    use this exec_query
                    recursion_or_compose()
        else:
            recursion_or_compose()

    def recursion_or_compose():
        if input_type is not none:
            recursive_select_complexity()
        else:
            compose()

    ... and so forth
\end{verbatim}

The \emph{vtsg.py} script creates a new object of the
\emph{Generator} class. 
The program iterates through all sink modules, selecting those
specified by the user, see
Sec.~\ref{sec:command line interface}, 
or all of them if the user does not specify. It subsequently
selects a compatible filter.  It then goes through all inputs.
The \verb|<input_type>| and \verb|<output_type>|
must be consistent with the ``Filter'' and ``Sink'' XML tags.
Then an exec query and complexities are selected
that are compatible with the currently selected sink module.

When VTSG has selected a set of modules, it begins composing
their code to generate the source code for a test case.
The process of composing modules to generate source code is
based on XML metadata tags.
After the imports and class definition declaration for the 
specific program
language, the ``Input'' metadata \verb|<code>| portion
is added to the test case.  
The ``Filtering'' metadata \verb|<code>| portion, plus its
\verb|<flaw type>| and safety indicator, are added to
the test case.  Next, the ``Sink'' metadata 
\verb|<code>| portion
is added to the test case.  Finally, the ``ExecQuery'' type is 
noted and the 
\verb|<code>| portion of the ``ExecQuery'' is added
to the 
test case.  The test case is written to a file.  
The location chosen for the file is
described in Sec.~\ref{sec:case directory structure}.
Section~\ref{sec:case file name} describes how VTSG names the file.

VTSG generates many different test cases, both with and without
flaws, with various control flow complexities.  After VTSG finishes
generating each vulnerability category, it displays how many safe (non-vulnerable)
and unsafe (vulnerable) test cases it produced.
VTSG generates hundreds of test cases in minutes.

VTSG is built to generate test cases with all consistent combinations
of modules for the flaw groups and flaws specified in the invocation.
If VTSG is invoked with flaw groups (\verb|-g|) or particular flaws (\verb|-f|),
only sinks satisfying those specified are used.
If no flaw group is specified, all flaw groups are used.  
If no flaws are specified, all flaws are used.

\label{sec:depth of complexities}
The depth command line option, \verb|-r| or \verb|--depth|, 
specifies the
most nested flow control complexities produced.
VTSG generates test cases with all complexities up to the depth
indicated.
For example, the default depth, 1, leads VTSG to generate all
test cases with no flow complexities and all test cases with 
one complexity.  The option \verb|-r 2| leads VTSG to generate
all cases with no complexities, all cases with one complexity, 
and all cases with two nested complexities.
See Sec.~\ref{sec:code complexities} for an example of three 
nested control flow complexities.

\subsection{Code Complexities}
\label{sec:code complexities}

In theory, a static analysis tool only needs to process a few lines of
code that
embody the vulnerability. In practice, a tool must analyze most 
of the program,
noting its control and data flows, to accurately track data and
determine the
conditions when the code with weaknesses may be executed.
Code complexities are constructs that may confuse static 
analysis tools.  
Each code complexity element can have many different attributes
associated with it.
They are combined and nested to create more realistic source code.
For example, the value 
of an expression may come from a constant, a single variable, 
some arithmetic 
combination, or the return value of a function call.  
Flow of control may be
influenced by loops, conditionals, and functions calls.  
Also, there could be many 
layers or depths of such nesting structures.  

The complexities currently defined for PHP
and \CSharp\ are listed in the appendixes for their languages.

Here is an example code complexity from \verb|cwe_89__I_shell_commands__F_no_|\\ 
\verb|filtering__S_select_from-concatenation_simple_quote__EQ_mysql| \\
\verb|__3-2.5-9-21a.cs|: \\
{\texttt
{\colorbox{yellow}{if((Math.Pow(4, 2)<=42))\{}}\\
\hspace*{2em}{\colorbox{green}{switch(6)\{}}\\
\hspace*{4em}{\colorbox{green}{case(6):}}\\
\hspace*{6em}{\colorbox{cyan}{Class\_489618 var\_489618 = new Class\_489618(tainted\_5);}}\\
\hspace*{6em}{\colorbox{cyan}{tainted\_6 = var\_489618.get\_var\_489618();}}\\
\hspace*{6em}tainted\_7 = tainted\_6;\\
\hspace*{6em}{\colorbox{green}{break;}}\\
\hspace*{4em}{\colorbox{green}{default:}}\\
\hspace*{6em}{\colorbox{green}{break;}}\\              
\hspace*{2em}{\colorbox{green}{\}}}\\
{\colorbox{yellow}{\}else\{}}\\
\hspace*{2em}{\colorbox{yellow}{\{\}}}\\
{\colorbox{yellow}{\}}}
}


The above has complexity depth 3, note \verb|__3| near the end of 
the file
name. They correspond to:\\
{\colorbox{yellow}{Level 1 is the conditional}} id = 2 with condition 5\\
\hspace*{2em}{\colorbox{green}{Level 2 is the switch statement}} id = 9\\
\hspace*{4em}{\colorbox{cyan}{Level 3 is the call of a method from a Class defined in a different file}}, id = 21.

\section{Template, Input, Filter, Sink, Exec\_Query, and Complexity Files}
\label{sec:source files}

These XML files are required for each language.
There 
are a few XML-specific caveats that must be paid attention to when 
creating these files. 
Table~\ref{tab:XML escapes} lists the symbols that may cause errors 
during the process and the XML equivalent replacement necessary to 
complete the
task without error.

\begin{table}[H]
\centering
\begin{tabular}{|c|l|}
\hline
\textbf{Character} & \textbf{Replacement} \\
\hline
 \verb|<| & \&lt; \\
\hline
 \verb|>| & \&gt; \\
\hline
 \verb|"| & \&quot; \\
\hline
 \verb|'| & \&apos; \\
\hline
 \verb|&| & \&amp; \\
\hline
\end{tabular}
\caption{Replacement sequences for characters that are treated 
in a special way in XML files.}
\label{tab:XML escapes}
\end{table}

VTSG uses Jinja to compose code. In addition to the above, Jinja recognizes
double-curly-brackets (\{\{ and \}\}) as introducing Jinja-specific variables and
controls.  Do not use pairs of curly brackets in your files, except for VTSG-related
variables.

Characteristics of modules and their information are stored in 
XML files.  
This section of the document describes the structure of each type of module and
the meaning
of each element and its tags.

Most of the file types have an example followed by
an explanation of what it does and what it generates.

Each language directory has one file of each name. That is,
one \verb|file_template.xml| file, one \verb|inputs.xml| file,
one \verb|filters.xml| file, one \verb|complexities.xml| file,
one \\ \verb|sinks.xml| file, and one \verb|exec_queries.xml| file.

All the files, except \verb|file_template.xml|, may have many
modules, that is alternate chucks of code, in them.
For example, \verb|inputs.xml|, Sec.~\ref{sec: input module}, typically has many input
modules.  Each module in \verb|inputs.xml| provides a method to get input from a
different source, such as command line options or hard-coded values.

\subsection{Maintain Indentation with INDENT ... DEDENT}
\label{sec:indent}

\subsubsection{Using INDENT ... DEDENT}
INDENT ... DEDENT sections may appear in any of the above files.  However, they
most often occur in \verb|file_template.xml| and \verb|complexities.xml| files.

VTSG using Jinja does a haphazard job of producing proper indentation.  Indentation
does not matter for many languages.  It is critical for Python, however.  To maintain
correct indentation, use INDENT and DEDENT lines in code chunks to indicate that
any code between those lines should be indented consistently.  For example,
\begin{verbatim}
def main():
INDENT
    {{local_var}}
    {{input_content}}
    {{filtering_content}}
    {{sink_content}}
    {{exec_queries_content}}
DEDENT
\end{verbatim}
All code produced from the statements between the INDENT/DEDENT lines is
consistently indented with the string defined in \verb|<indent>| \verb|</indent>|,
which appears in \\ \verb|file_template.xml|, see Sec.~\ref{sec: file template}.
That string is typically four spaces.

INDENT sections may be nested.  For example, here is a sink module with code that
needs additional indentation.
\begin{verbatim}
        print(f'file "{ {{in_var_name}} }" ', end='')
        {{flaw}}
        if os.path.exists({{in_var_name}}):
	INDENT
            print('exists')
	DEDENT
        else:
	INDENT
            print('does not exist')
	DEDENT
\end{verbatim}
If the INDENT lines were not included, VTSG produces the following code (slightly
edited for presentation).
\begin{verbatim}
def main():
    tainted_0 = input()
    tainted_1 = tainted_0
    
        # No filter (sanitization)
        tainted_1 = tainted_0
            
    
        print(f'file "{ tainted_1 }" ', end='')
        #flaw
        if os.path.exists(tainted_1):
            print('exists')
	else:
            print('does not exist')
\end{verbatim}
Notice that the indentation is not consistent. This is not valid Python code.
With INDENT lines, VTSG produces the following, which is valid Python.
\begin{verbatim}
def main():
    tainted_0 = input()
    tainted_1 = tainted_0

    # No filter (sanitization)
    tainted_1 = tainted_0


    print(f'file "{ tainted_1 }" ', end='')
    #flaw
    if os.path.exists(tainted_1):
        print('exists')
    else:
        print('does not exist')
\end{verbatim}


\subsubsection{Details of INDENT ... DEDENT}

Here are details of using INDENT and DEDENT.

VTSG processes code within INDENT sections line by line. No semantic parsing or
analysis is done.

A section to be fixed is indicated by a line beginning with \verb|INDENT|, possibly
with leading whitespace. The end of the section is indicated by a line beginning with
\verb|DEDENT|, again possibly with leading whitespace. Any text after INDENT or
DEDENT to the end of the line is ignored.

Indent sections may be nested.

INDENT and DEDENT lines are removed.
For lines within an INDENT ... DEDENT section,
\begin{itemize}
\item first, any leading whitespace is removed, and
\item second, indent is added for each nested INDENT ... DEDENT section this
  is in if the line is not empty.
\end{itemize}

Here is a convoluted example to illustrate the fine points.  Suppose this is the
code generated by composing the modules.
\begin{verbatim}
      if Condition:
INDENT            text after INDENT is ignored
           line 1
               while not True:
          INDENT
           line 3 - INDENT not at the beginning is ignored
     DEDENT

                  line above is empty
        DEDENT
        line 5
\end{verbatim}
If the indent string is specified as \verb|<indent>..,</indent>|, the following is
the result. (Note: typically, the indent is four spaces. The preceding string with
periods and a comma is only for example clarity.)
\begin{verbatim}
      if Condition:
..,line 1
..,while not True:
..,..,line 3 - INDENT not at the beginning is ignored

..,line above is empty
        line 5
\end{verbatim}

Note: because \emph{all} leading whitespace is removed from lines in indent sections,
using INDENT ... DEDENT anywhere means that every indentation must be indicated
with INDENT ... DEDENT lines.

We chose ``DEDENT'' because it is used in Python's grammar description.


\subsection{File Template}
\label{sec: file template}

\begin{verbatim}
<template type="" name="">
    <file_extension></file_extension>
    <comment>
        <open></open>
        <close></close>
        <inline></inline>
    </comment>
    <syntax>
        <statement_terminator></statement_terminator>
        <indent></indent>
    </syntax>
    <namespace></namespace>
    <variables prefix="" import_code="using {{import_file}};">
        <variable type="" code="" init=""/>
    </variables>
    <imports>
        <import></import>
    </imports>
    <code></code>
</template>
\end{verbatim}

\begin{itemize}
    \item name: Programming language name, e.g., PHP, Csharp,
    or Python. This appears in the manifest.
    It is also the name of the subdirectory under the TestSuite directory
    where all the generated test cases are placed.

    \item file\_extension: Extension of the generated files.

    \item comment: Strings indicating comments.
    \begin{itemize}
        \item open: string to begin a comment, which may span many lines
        \item close: string to end a comment, which may span many lines
        \item inline: string to begin a one-line comment
    \end{itemize}
    
    \item syntax: Other language-specific syntax.
    \begin{itemize}
        \item statement\_terminator: string to show the end of 
        a statement.
        This is semicolon
        \verb|<statement_terminator>;</statement_terminator>|
        in PHP, C, Java, \\ and \CSharp. Python does not have 
        a terminator, so this is the empty string: \\
        \verb|<statement_terminator></statement_terminator>|.

        \item indent: string used to indent code, see Sec.~\ref{sec:indent}.
          This is typically four spaces, but
          can be any string.
    \end{itemize}
    
    \item namespace: Namespace name, if applicable

    \item variables: Information about variable names and types and how
    to include libraries.
    \begin{itemize}
        \item prefix: Any prefix required for variable names.
        \$ for PHP.
        Leave it empty if not required.

        \item import\_code: Code to include a library. The code
        should have the placeholder \verb|{{import_file}}|.
        For example, \verb|#include <{{import_name}}>|.
        
        \item variable: Defines each variable type and how it will 
        be used.
        \begin{itemize}
        \item type: Names the type. This string 
        does not appear in the test case code.  It tells VTSG
        the type of variable that is being used.  The input\_type
        and output\_type in Input, Filter, and Sink modules use
        this string.

        \item code: A piece of code declaring the type of the variable. For 
        some languages, such as PHP and Python, this field can be blank. 
        This value takes the variable type when being declared 
        (for example, \verb|string myString;|).  In this case, ``string''
        is the value put in this attribute.

        \item init: Value assigned when this type of variable is initialized.
        VTSG uses this value when declaring all variables.
        \end{itemize}
    \end{itemize}
    
    \item code: the template code. It should contain the 
    following placeholders:
    \begin{itemize}
        \item comments: This is replaced by comments in the selected input,
        filter, sink, and exec query modules.  This is intended to
        describe the variants, options, and use of this test case. 
        
        \item license: This is replaced by the contents of the
        \verb|file_rights.txt| file.  This is intended to hold
        authors' names, usage and copyrights, contact information, 
        etc.
        
        \item stdlib\_imports:  This is a placeholder for
        \emph{all} imports for the generated program

        \item namespace\_name:  Used if the language requires it

        \item main\_name:  Name of the main class

        \item local\_var:  Location for local variables (required)

        \item input\_content:  Location for the Input (required)

        \item filtering\_content:  Location for complexities, if any, along
        with the Filter

        \item sink\_content:  Location for the Sink

        \item exec\_queries\_content:  Location for the ExecQuery

        \item static\_methods:  Location for the static functions.
    \end{itemize}
\end{itemize}


\subsection{Attributes Shared By Modules}
\label{sec:shared attributes}

Many kinds of modules use the same attributes.  Instead of repeating explanation of
these attributes, they are here.

\subsubsection{Module Description in Path and Dir Tags}
\label{sec:module description}

Within the \verb|<path>| keywords, modules may have one or more \verb|<dir>| tags.
These tags provide the descriptions of the module that is used in the file name, see
Sec.~\ref{sec:case file name}.  For example, when the key word in a selected input
module is ``file'', the file name will contain \verb|..._I_file_...|, where ``I''
indicates the input module selected.

If a module has more than one \verb|<dir>| tag, the strings are joined with dashes.
For example, if a sink has
\begin{verbatim}
            <path>
                <dir>select_from</dir>
                <dir>concatenation_simple_quote</dir>
            </path>
\end{verbatim}
then cases using that module will have file names containing \\
\verb|_S_select_from-concatenation_simple_quote_|.

Note: we cannot think of any reason why it is better to give multiple description
strings instead of just one string.  But the functionality is provided in VTSG, and
some modules use it.


\subsubsection{Module Comment}
\label{sec:module comment}

If a sample module has a comment string, it is added to the
\verb|{{comments}}| area given in the file template, Sec.~\ref{sec: file template}.
This informs the user about the purpose or structure of the input, filter, sink, and
exec query modules included.  Below is an example comment string.

\begin{verbatim}
            <comment>sink: check if a file exists</comment>
\end{verbatim}

Any case using that module will have
\begin{verbatim}
sink: check if a file exists
\end{verbatim}
in the comments area.


\subsubsection{Needed Imports}
\label{sec:module import}

Sometimes the use of code requires some library to be imported or used.  This is
indicated with names given in \verb|<import></import>| directives within
\verb|<imports></imports>| sections.

Code statements are synthesized from the \verb|import_code| in the file template and
the name or names given here.


\subsubsection{Marking Modules as Safe and Unsafe}
\label{sec:safe or unsafe}

Using some modules in a program for certain flaws may make them safe or may make them
unsafe.  For instance, prepared SQL statements are always safe from SQL injection
vulnerabilities.  In contrast using a broken cryptographic algorithm is always
unsafe, regardless of how any user input is filtered.  Similarly certain
hard-coded inputs may always make a program safe from certain flaws, and some filters
may make a program safe from certain flaws for any user input.

Input, filter, and sink modules can be marked as always safe or always unsafe using
\verb|safe="1"| or \verb|unsafe="1"|.  Modules may be always safe or always unsafe
(or neither) for some flaws and have different safety attributes for other
flaws.

Exec query modules may be marked as always safe.  (No exec query module can make the
program unsafe.)

A generated program is not safe if any of the selected input, filter, or sink modules
are always unsafe, that is \verb|unsafe="1"|.  A program is safe if any of the
selected input, filter, sink, or exec query modules is always safe, that is
\verb|safe="1"|, \emph{and} none
are unsafe.  The filter module must be executed to be considered.  In other words, if
a complexity never executes the filter, then the filter's safe or unsafe marking is
ignored.  Table~\ref{tab:selection safe algorithm} expresses this as a table.

\begin{table}[H]
\centering
\begin{tabular}{c|c|c|}
  & \makecell{Any module \\ has safe="1"}
  & \makecell{No module is \\ always safe} \\
\hline
\makecell{Any module \\ has unsafe="1"}  & not safe & not safe \\
\hline
\makecell{No module is \\ always unsafe} &   safe   & not safe \\
\hline
\end{tabular}
\caption{Decision table for whether a selection of modules is considered safe or unsafe.}
\label{tab:selection safe algorithm}
\end{table}

The Code Complexity Modules, Sec.~\ref{sec: complexity modules}, explains when a
filter may never be executed.


\subsection{Input Modules}
\label{sec: input module}

The \verb|inputs.xml| file has one or more ``sample'' input modules.  Each module
provides one way for the generated program to get input.

\begin{verbatim}
<sample>
    <path>
        <dir></dir>
    </path>
    <comment></comment>
    <flaws>
        <flaw flaw_type="" [safe=""] [unsafe=""]/>
    </flaws>
    <imports>
        <import></import>
    </imports>
    <code></code>
    <input_type></input_type>
    <output_type></output_type>
</sample>
\end{verbatim}

\begin{figure}[htb]
  \includegraphics{fig_Input_file.png}
  \caption{Example Input module.  Instantiated at line 14 of 
  Fig.~\ref{fig:example main file}.}
  \label{fig:example input file}
\end{figure}

\begin{itemize}
    \item flaw: vulnerability categories where the sample can be used, that is, the
      flaws that the input is compatible with.  Also whether it is always safe or
      always unsafe for that flaw.  See Sec.~\ref{sec:safe or unsafe}
      for more details.  If this input is generic and compatible with all
      types of vulnerabilities, put ``default'' as the \verb|flaw_type|.

    \item input\_type: this string is placed in the manifest.  It has no other
      function in VTSG.

    \item output\_type: the type of output.  The variable generated with the
      placeholder \\ \verb|{{out_var_name}}| in the code will be that type.

    \item code: The source code of an input. It should contain the placeholder \\
      \verb|{{out_var_name}}|.  That placeholder will be replaced by the variable
      name used in the Filter and Sink.  Do not declare this variable.
\end{itemize}

The case generated from the example Input in 
Fig.~\ref{fig:example input file}
takes an argument from the command line as Input.  
The input string can be either safe or unsafe, depending on user input.

\subsection{Filter Modules}

All filter modules are in the \verb|filters.xml| file.

\begin{verbatim}
<sample>
    <path>
        <dir></dir>
    </path>
    <comment></comment>
    <flaws>
        <flaw flaw_type="" [safe=""] [unsafe=""]/>
    </flaws>
    <imports></imports>
    <code></code>
    <input_type></input_type>
    <output_type></output_type>
</sample>
\end{verbatim}

\begin{figure}[htbp]
  \includegraphics[width=\linewidth]{fig_Filter_file.png}
  \caption{Example Filter module. Instantiated in lines 19--27 of Fig.~\ref{fig:example aux file}.}
  \label{fig:example filter file}
\end{figure}

\begin{itemize}
    \item input\_type: the input type of the filter. The variable 
    generated with
    the placeholder \verb|{{in_var_name}}| will be that type.
    Declarations of \verb|variable| in the File Template give
    available types, see Sec.~\ref{sec: file template}.

    \item output\_type: the output type of the filter.  The variable generated
    with the placeholder \verb|{{out_var_name}}| will be that type. \\
    Tip: To generate a test without Filter, assign \verb|in_var_name| to
    \verb|out_var_name| and make the input\_type and output\_type \verb|nofilter|.
    This passes the variable from the Input directly to the Sink.

    \item flaw: vulnerability categories where the filter can be used, that is, the
      flaws that the filter is compatible with.  Also whether it is always safe or
      always unsafe for that flaw.  See Sec.~\ref{sec:safe or unsafe}
      for more details.  If this filter is generic and compatible with all
      types of vulnerabilities, put ``default'' as the \verb|flaw_type|.

    \item code: The source code of an filter. It should contain the placeholders
    \\
    \verb|{{in_var_name}}| and \verb|{{out_var_name}}|.  Those placeholders will 
    be replaced by the variable names used in the Input and Sink.  Do not declare 
    these variables.
\end{itemize}

The example Filter file in Fig.~\ref{fig:example filter file} makes sure
the Input contains only a number.  
The flag safe is 1, because you cannot cause an SQL Injection 
(CWE 89) with only numbers.


\subsection{Sink Modules}
\label{sec:sink modules}

All sink modules are in the language's \verb|sinks.xml| file.

\begin{verbatim}
<sample>
    <path>
        <dir></dir>
    </path>
    <flaw_type flaw_group=""></flaw_type>
    <safety safe="" unsafe=""/>
    <comment></comment>
    <imports>
        <import></import>
    </imports>
    <code></code>
    <input_type></input_type>
    <exec_type></exec_type>
</sample>
\end{verbatim}

\begin{itemize}
    \item flaw\_type: the flaw\_group is a general category of vulnerability.
    Generated test cases are placed under the flaw group subdirectory, then
    in the flaw type subdirectory under that.  If the flaw\_group is missing or
    empty, flaw type subdirectories are created immediately under the language
    directory.
    The user can limit
    cases generated to certain flaw groups with the \verb|-g| command
    line option or certain flaws with the \verb|-f| option.
    
    \item input\_type: the input type of the sink. The variable
    generated with the placeholder \verb|{{in_var_name}}| will be 
    that type.  If the sink does not
    require an input, this type should be \verb|none|. The code 
    should not contain
    the placeholder \verb|{{in_var_name}}|.
    Declarations of \verb|variable| in the File Template give
    available types, see Sec.~\ref{sec: file template}.

    The input type specifies the kind of data this sink needs from the filter (or
    from the input).  VTSG only selects filters whose output types are the same as
    this input type.  If the filter is ``nofilter'', then VTSG selects inputs whose
    output types are the same as this input type.

    \item exec\_type: link a sink to the exec queries.  It must have 
    the type of
    an ExecQuery. If it does not require an ExecQuery, 
    exec\_type should be \verb|none|.

    \item safety: whether the sink is always safe or always unsafe.  For instance, a
      deprecated function may be marked (always) unsafe.
      See Sec.~\ref{sec:safe or unsafe}
      for more details.
    
    \item code: The source code of a sink. It should contain the placeholder
    \verb|{{in_var_name}}|.  The placeholder will be replaced by the variable
    name used in the Filter.  Do not declare this variable.

    The placeholder \verb|{{flaw}}| indicates that the next line is the location
    of the flaw.  In other words, if this case is unsafe, the manifest reports a
    flaw at the number of the line following this.  In generated unsafe cases,
    \verb|{{flaw}}| is replaced with the one-line comment string,
    see Sec.~\ref{sec: file template}, and ``flaw''.
    It does not appear in generated safe cases.
\end{itemize}

\begin{figure}[htbp]
  \includegraphics[width=\linewidth]{fig_Sink_file.png}
  \caption{Example Sink module. Instantiated at line 23 of Fig.~\ref{fig:example main file}.}
  \label{fig:example sink file}
\end{figure}

The Sink example in Fig.~\ref{fig:example sink file} 
concatenates the filtered string with an SQL query.  This block
of code can only be used for SQL Injection.  Whether or not it is
vulnerable depends on the input string.


\subsection{Exec\_Query Modules}

All query execution modules are in the language's \verb|exec_query.xml|
file.

\begin{verbatim}
<exec_query type="" safe="">
    <path>
        <dir></dir>
    </path>
    <comment></comment>
    <imports>
        <import></import>
    </imports>
    <code></code>
</exec_query>
\end{verbatim}

\begin{itemize}
    \item type: the type of the ExecQuery. This is used in the
    \verb|exec_type| tag of the Sink to link them together during
    generation process.  The type should only contain letters, numerals, and
    underscore (``\_'').\\
    Languages currently available for VTSG V3 support many database management
    systems, including ORACLE, MySQL, MSSQL, PostgreSQL, SQLite, and XPATH.
    The syntax of each ExecQuery must be
    compatible with its associated database system language.
    
    \item safe: whether the ExecQuery always makes the case safe.
    See Sec.~\ref{sec:safe or unsafe} for more details.

    \item code: The source code of a query. It does not contain placeholders.
    It should be linked to the corresponding variable from the Sink. The linking is done through the "exec\_type" attributes within the XML files.
\end{itemize}


\begin{figure}[htbp]
  \includegraphics[width=\linewidth]{fig_Exec_Query_file.png}
  \caption{Example Exec\_Query module. Instantiated in lines 25--39 of
    Fig.~\ref{fig:example main file}.}
  \label{fig:example exec-query file}
\end{figure}

The block of code in the Exec\_Query example, 
Fig.~\ref{fig:example exec-query file}, executes the SQL query, used 
for database management
systems, including MySQL, Oracle, PostgreSQL, and SQLite.  This example is
vulnerable.  If a non-vulnerable execution of an SQL query is required,
use an SQL prepared statement.


\subsection{Test Condition and Code Complexity Modules}

All test condition and code complexity modules are in the 
language's \verb|complexities.xml| file.  This file has
a \verb|<root>| with one \verb|<conditions>| part and one
\verb|<complexities>| part.
All condition modules are inside \verb|<conditions>|.  All
complexity modules are inside \verb|<complexities>|.

\begin{verbatim}
<root>
    <conditions>
        <condition ...>
            ...
        </condition>
        ....
    </conditions>
    <complexities>
        <complexity ...>
            ...
        </complexity>
        ....
    <complexities>
\end{verbatim}

\subsubsection{Test Condition Modules}
\label{sec: condition modules}

\begin{verbatim}
<condition id="">
    <code></code>
    <value></value>
</condition>
\end{verbatim}

\begin{itemize}
    \item id: string indicating this condition.  Appears in the test case
      file name.  Typically this is a number.

    \item code: the source code of the conditional test.

    \item value: either \verb|<value>True</value>| or
        \verb|<value>False</value>| depending on \\
        whether the code evaluates to true or false.
\end{itemize}

\begin{figure}[htbp]
  % width makes the text about the same size as text in Complexity_file_while
  \includegraphics[width=2.4in]{fig_Complexity_file_test.png}
  \caption{Example test condition module.  Instantiated in
    Fig.~\ref{fig:example main file}, line 16.}
  \label{fig:example complexity-test file}
\end{figure}


\subsubsection{Code Complexity Modules}
\label{sec: complexity modules}

\begin{verbatim}
<complexity id="" type="" group="" executed="" in_out_var="i" 
                       need_condition="" indirection="" need_id="">
    <code></code>
    <body></body>
</complexity>
\end{verbatim}

\begin{itemize}
    \item id: string indicating this complexity.  Appears in the test case
      file name.  Typically this is a number.

    \item type: Supported types are: \verb|if|, \verb|switch|, \verb|goto|, 
    \verb|for|, \verb|foreach|, \verb|while|, \\
    \verb|function|, and \verb|class|.
    If the type is \verb|class|, source code in the \verb|<body></body>| is placed in
    an additional file that is created the this case.
    Invocation statements are generated for \verb|function| and \verb|class| types.
    An extra variable is created for \verb|foreach| types (with group \verb|loops|).
    No other type has any effect on VTSG.

    \item group: Supported groups are: \verb|conditionals|, \verb|jumps|, 
    \verb|loops|, \verb|functions|, and \\ \verb|classes|.
    No group, other than \verb|loops|, has any effect on VTSG.

    \item executed: whether the placeholder will be executed or not. Four 
    values are allowed:
    \begin{itemize}
        \item 0: Never executed
        \item 1: Always executed
        \item condition:  Executed if the condition is true
        \item not\_condition:  Executed if the condition is false
    \end{itemize}
    Table~\ref{tab:execution examples} gives example code for each value.

    \begin{table}[H]
    \centering
    \begin{tabular}{|r|l|}
    \hline
      \makecell{Value of \\ executed}
      & example code \\
    \hline
    0 &
    \begin{minipage}{3in}
    \begin{verbatim}


        switch(6) {
          case(6):
            break;
          default:
            {{ placeholder }}
            break;
        }
    \end{verbatim}
    \end{minipage}
    \\
    \hline
    1 &
    \begin{minipage}{3in}
    \begin{verbatim}


        switch(6) {
          case(6):
            {{ placeholder }}
            break;
          default:
            break;
        }
    \end{verbatim}
    \end{minipage}
    \\
    \hline
    condition &
    \begin{minipage}{3in}
    \begin{verbatim}


      if ({{ condition }}) {
            {{ placeholder }}
        } else {
            {}
        }
    \end{verbatim}
    \end{minipage}
    \\
    \hline
    not\_condition &
    \begin{minipage}{3in}
    \begin{verbatim}


        if ({{ condition }}) {
            {}
        } else {
            {{ placeholder }}
        }
    \end{verbatim}
    \end{minipage}
    \\
    \hline
    \end{tabular}
    \caption{An example of code for each value of executed.}
    \label{tab:execution examples}
    \end{table}

    \item in\_out\_var: whether the variable (from the Input) will be used or
    transformed in the Complexity before being used in the Filter.  If the
    variable is neither used nor transformed, do not use this attribute.
    Three values are allowed:
    \begin{itemize}
        \item in: the variable is used before the placeholder 
        \item out: the variable is used after the placeholder
        \item traversal: the variable is used in the placeholder
    \end{itemize}
    If this attribute is used, the code should contain the following 
    placeholders: \\
    \verb|{{in_var_name}}|, \verb|{{out_var_name}}|, and \verb|{{var_type}}|.

    \item need\_condition: ``1'' if this complexity needs a condition.  By default
      this complexity is combined with conditions,
      see Sec.~\ref{sec: condition modules}, if \verb|executed| is \verb|condition|
      or \verb|not_condition|. (optional)

    \item indirection: ``1'' if the code is split into two chunks (call and
    declaration) or calls a function.  The body tag should be present when 
    calling a function.

    \item need\_id: ``1'' if the code has a placeholder, \verb|{{id}}|,
    to generate a unique ID for the Complexity.  This ID to generate 
    a label, a parameter, or a function name in a nested
    context.

    \item code: the source code of the Complexity.  Code or 
    body should contain \\ \verb|{{placeholder}}|
    where the Filter is inserted.  It may also contain
    \verb|{{condition}}| where the Condition
    is inserted.

    \item body: additional source code not in the main execution flow,
    e.g., functions or classes.  This code is placed in a separate file
    of the case if the type is \verb|class|. (optional)
\end{itemize}


\begin{figure}[htbp]
  \includegraphics[width=4in]{fig_Complexity_file_while.png}
  \caption{Example Complexity module with a {\texttt while} loop.  Instantiated in 
    Fig.~\ref{fig:example main file}, lines 16--21.}
  \label{fig:example complexity-while file}
\end{figure}

\begin{figure}[htbp]
  \includegraphics[width=\linewidth]{fig_Complexity_file_method.png}
  \caption{Example Complexity module with a method invocation.  
  The \texlangle code\texrangle\  part is instantiated in
  Fig.~\ref{fig:example main file}, lines 18 and 19. 
  The \texlangle body\texrangle\  part is instantiated in
  Fig.~\ref{fig:example aux file}.}
  \label{fig:example complexity-method file}
\end{figure}

VTSG can use several complexities in one test case.
The example in Sec.~\ref{sec:generated files} has two types of
Complexity: a control flow complexity and a data flow complexity. 
The control flow complexity specification is in 
Fig.~\ref{fig:example complexity-while file}.  It is instantiated in 
lines 16--21 of
Fig.~\ref{fig:example main file}.  Line 16 is the instantiation of 
the control flow condition
specified in Fig.~\ref{fig:example complexity-test file}.

The data flow complexity is a method call within the \verb|while| loop.  
The specification
is in Fig.~\ref{fig:example complexity-method file}.  
The \verb|<code>|
part is instantiated in lines 18 and 19 of 
Fig.~\ref{fig:example main file}. 
The \verb|<body>| part is instantiated in
Fig.~\ref{fig:example aux file}.


\section{Generated Test Case File Names}

This section describes what the
names of test case files mean.

VTSG creates directories and subdirectories for the test cases that it generates.
The directory structure is described in
Sec.~\ref{sec:case directory structure}.

%\subsection{Test Case File Names}
\label{sec:case file name}

VTSG names test case files as
\verb|FLAW__I_INPUT__F_FILTER__S_SINK__EQ_EXEC_| \\
\verb|QUERY__NBCPLX-CPLX1-CPLX2.CONDx.EXT|
\begin{itemize}
    \item FLAW: Flaw type, e.g. CWE\_89, BF, or STR30-PL
    \item INPUT:  Input description (optional)
    \item FILTER:  Filter description (optional)
    \item SINK:  Description of the critical function
    \item EXEC\_QUERY:  ExecQuery description (optional)
    
    \item NBCPLX:  The number of complexities. Each complexity has the tags 
    \begin{itemize}
        \item CPLX1, CPLX2, \ldots: ID given in 
            code complexity modules, 
            see Sec.~\ref{sec: complexity modules}.
            Tables \ref{tab:complexity IDs for CSharp} and
            \ref{tab:complexity IDs for PHP} in the language appendixes list
            complexity IDs used in \CSharp\ and PHP.
            (optional)
        \item COND: ID given in test condition modules,
            see Sec.~\ref{sec: condition modules}.
            Tables \ref{tab:condition IDs for CSharp} and
            \ref{tab:condition IDs for PHP} list condition
            IDs used in \CSharp\ and PHP. (optional)
    \end{itemize}
    \item x: Sequence of the file within the test.  If the test consists of just one
      file, there is no sequence letter.  If the test consists of more than one file,
      that is, when the complexity \verb|type| is \verb|class|,
      see Sec.~\ref{sec: complexity modules},
      the main file is ``a'', and other files, such as classes, are ``b'', ``c'',
      ``d'', etc.
    \item EXT: file extension, given in the \verb|file_template.xml| file, see
    Sec.~\ref{sec: file template}.
\end{itemize}

See the next section, \ref{sec:generated files}, for an example of the file name of a
generated test case and the various pieces.

File names reflect the entire case, not just the code in a 
particular file.  If a case consists of more than one file, as in the
example in Sec.~\ref{sec:generated files}, all files have
identical names, except for the final sequence letter.

\section{Example Test Case}
\label{sec:generated files}

\begin{figure}[htbp]
  \includegraphics[width=\linewidth]{fig_example_code1.png}
  \caption{Main file of example. Line 14 instantiates input code from 
    Fig.~\ref{fig:example input file}. Lines 16--19 instantiates complexity code from 
    Fig.~\ref{fig:example complexity-while file}. Line 16 instantiates condition code from
    Fig.~\ref{fig:example complexity-test file}.  Lines 18 and 19 instantiate code from the
    \texlangle code\texrangle\  part of Fig.~\ref{fig:example complexity-method file}.
    Line 23 instantiates critical preparation code from Fig.~\ref{fig:example sink file}.
    Lines 25--39 instantiate query execution code from Fig.~\ref{fig:example exec-query file}.
  }
  \label{fig:example main file}
\end{figure}

\begin{figure}[htbp]
  \includegraphics[width=0.85\linewidth]{fig_example_code2.png}
  \caption{Auxiliary file of example.  It instantiates code from the
    \texlangle body\texrangle\ part of Fig.~\ref{fig:example complexity-method file}.
    Lines 19--27 instantiate filter code
    from Fig.~\ref{fig:example filter file}.}
  \label{fig:example aux file}
\end{figure}

This section has an example test case.  The case consists of 
two files.
Fig.~\ref{fig:example main file} is the main file of the example.  
Fig.~\ref{fig:example aux file} is an auxiliary class file.  
The code in
the main file invokes the class at line 18.

The name of the main file is
\verb|CWE_89__I_shell_commands__F_func_preg_| \\
\verb|match-only_numbers__S_select_from-concatenation_simple_quote__| \\
\verb|sql_server__2-11.7-20a.cs|.
The name of the class file, Fig.~\ref{fig:example aux file}, is
identical, except for the
file letter ``b'' instead of ``a'' at the end.

Using Sec.~\ref{sec:case file name} and the extension, cs, which shows that it is a
\CSharp\ file, the file name is read as follows:

\noindent CWE 89: Improper Neutralization of Special Elements used in an SQL Command
('SQL Injection') \cite{CWE89}

\noindent The input comes from shell\_commands, see specification in
Fig.~\ref{fig:example input file}.

\noindent The filter is func\_preg\_match-only\_numbers,
Fig.~\ref{fig:example filter file}.

\noindent The sink is select\_from-concatenation\_simple\_quote\_\_sql\_server,
Fig.~\ref{fig:example sink file}.

\noindent The next part, 2-11.7-20, means this has two complexities.
Tables~\ref{tab:complexity IDs for CSharp} and
\ref{tab:condition IDs for CSharp} help us decode them.
The first, outer complexity is 11 with condition 7. 11 is \verb|while|,
Fig.~\ref{fig:example complexity-while file}, 
with condition 7 meaning \verb|Math.Sqrt(42)<=42|, which always evaluates to true,
Fig.~\ref{fig:example complexity-test file}.
The second, inner complexity is 20 meaning the sink code is executed in the class body,
Fig.~\ref{fig:example complexity-method file}.

\noindent ``a'' means this is the main file.



\section{Acknowledgments}

Bertrand Stivalet and Aurelien Delaitre, from the NIST SAMATE team, designed the
architecture of the VTSG project and managed its implementation.  The project was
implemented by students from TELECOM Nancy: Jean-Philippe Eisenbarth, Valentin
Giannini, and Vincent Noyalet.  We thank Terry Cohen for her comments and
suggestions, which improved this report.  We also thank Elizabeth Fong and
Charles de Oliveira for their contributions to this report.
% On 12 June 2020, Terry Cohen wrote the following [[much editing by PEB]]
% Bertrand and Liz wrote the original document. I did some quick grammatical editing. Liz submitted it, but Barbara wanted it to cover both C# and PHP. The original did not have both. Because of the SATE V and CSIAC projects, this was put on hold.
%
% As I revised the NIST IR, I went through all of the instructions to verify the PHP statistics in the paper that Bertrand and Liz published and to generate C# statistics. I discovered errors and/or missing commands in the instructions. Charles pointed out the missing instructions.
%
% I recommend crediting Bertrand and Liz, because they started documentation.  I
% wrote the new document covering both C# and PHP.
%
% I added Charles because he helped me find the missing commands.  At the time, he was indifferent to being listed as a co-author.  We were using it as a learning exercise.
%
% In summary, I recommend that Bertrand, Liz, you, and your team be listed as co-authors.


% references section
\addcontentsline{toc}{section}{References}
\bibliographystyle{techpubs}
\bibliography{vulTestSuiteGenV3}

\newpage

%%%%%%%%%%%%%%%%%%%%%%%%%%%%%%%%%%%%%%%%%%%%%%%%%%%%%%%%%%%%%%%%%%%%%
%
%           APPENDIX
%
%%%%%%%%%%%%%%%%%%%%%%%%%%%%%%%%%%%%%%%%%%%%%%%%%%%%%%%%%%%%%%%%%%%%%

% make the Appendix begin on a new page (not a new column) even in
% two-column mode
\clearpage

\appendix

\section{\CSharp\ language}
\label{sec:CSharp language}

This appendix documents the flaws, flaw groups, conditions, and complexities
currently in the
\CSharp\ language files.

The following flaws are currently defined for this language:
\begin{itemize}
    \item SQL Injection (CWE-89)
    \item XPath Injection (CWE-91)
    \item LDAP Injection (CWE-90)
    \item OS Command Injection (CWE-78)
    \item Path traversal (CWE-22)
    \item Information Leak Through Error Message (CWE-209)
    \item Storing Password in Plain Text (CWE-256)
    \item Use of Insecure Cryptographic Algorithm (CWE-327)
    \item NULL Pointer Dereference (CWE-476)
\end{itemize}

The flaws are in the following groups:
\begin{itemize}
    \item OWASP\_a1 has CWE\_78, CWE\_89, CWE\_89, CWE\_89, CWE\_90, CWE\_91,
      CWE\_91, and CWE\_91.
    \item OWASP\_a2 has CWE\_256.
    \item OWASP\_a4 has CWE\_22.
    \item OWASP\_a5 has CWE\_209.
    \item OWASP\_a6 has CWE\_327
    \item OWASP\_a9 has CWE\_476.
\end{itemize}

Here are the complexities currently available in \CSharp.
We explain the concept of code complexities in Sec.~\ref{sec:code complexities} and
the format of complexity modules in Sec.~\ref{sec: complexity modules}.
This very brief description is to remind the reader of the complexity.
See the \verb|complexity.xml| file for the specific code.

\begin{table}[H]
\centering
\begin{tabular}{|r|l|}
\hline
\textbf{ID} & \textbf{Complexity} \\
\hline
 1 & if condition code \\
\hline
 2 & if condition code else \\
\hline
 3 & if condition else code \\
\hline
 4 & if condition code else if not condition \\
\hline
 5 & if condition else if not condition code \\
\hline
 6 & if condition code else if not condition else \\
\hline
 7 & if condition else if not condition code else \\
 \hline
 8 & if condition else if not condition else code \\
\hline
 9 & switch code executed \\
\hline
10 & switch code not executed \\
\hline
11 & while code \\
\hline

12 & do code while \\
\hline
13 & for code \\
\hline
14 & foreach code \\
\hline
15 & goto code not executed \\
\hline
16 & goto code executed \\
\hline
17 & function body executes code \\
\hline
18 & input passed via function then code \\
\hline
19 & code then output passed via function \\
\hline
20 & class body executes code \\
\hline
21 & input passed via class then code \\
\hline
22 & code then output passed via class \\
\hline
\end{tabular}
\caption{IDs and Code Description of Complexities Defined for \CSharp}
\label{tab:complexity IDs for CSharp}
\end{table}

Here are the conditions currently available to be used in code complexities.
We explain condition modules in Sec.~\ref{sec: condition modules}.
Table~\ref{tab:condition IDs for CSharp} shows the ID, the code, and whether it
always evaluates to true or false.

\begin{table}[H]
\centering
\begin{tabular}{|r|l|l|}
\hline
\textbf{ID} & \textbf{Code} & \textbf{Value} \\
\hline
1 & \verb|1==1| & True \\
\hline
2 & \verb|1==0| & False \\
\hline
3 & \verb|4+2<=42| & True \\
\hline
4 & \verb|4+2>=42| & False \\
\hline
5 & \verb|Math.Pow(4, 2)<=42| & True \\
\hline
6 & \verb|Math.Pow(4, 2)>=42| & False \\
\hline
7 & \verb|Math.Sqrt(42)<=42| & True \\
\hline
8 & \verb|Math.Sqrt(42)>=42| & False \\
\hline
\end{tabular}
\caption{IDs, Code, and Value to Which it Evaluates of Conditions Defined for
  \CSharp}
\label{tab:condition IDs for CSharp}
\end{table}


\section{PHP language}
\label{sec:PHP language}

This documents the flaw, flaw group, conditions, and complexities currently in the
PHP language files.

The following flaw is currently defined for this language:
\begin{itemize}
    \item SQL Injection (CWE-89)
\end{itemize}

The flaw is in the following group:
\begin{itemize}
    \item OWASP\_injection has CWE\_89.
\end{itemize}

Here are the complexities currently available in PHP.
We explain the concept of code complexities in Sec.~\ref{sec:code complexities} and
the format of complexity modules in Sec.~\ref{sec: complexity modules}.
This very brief description is to remind the reader of the complexity.
See the \verb|complexity.xml| file for the specific code.

\begin{table}[H]
\centering
\begin{tabular}{|r|l|}
\hline
\textbf{ID} & \textbf{Complexity} \\
\hline
 1 & if condition code \\
\hline
 2 & if condition code else \\
\hline
 3 & if condition else code \\
\hline
 4 & if condition code else if not condition \\
\hline
 5 & if condition else if not condition code \\
\hline
 6 & if condition code else if not condition else \\
\hline
 7 & if condition else if not condition code else \\
 \hline
 8 & if condition else if not condition else code \\
\hline
 9 & switch code executed \\
\hline
10 & switch code not executed \\
\hline
11 & while code \\
\hline

12 & do code while \\
\hline
13 & for code \\
\hline
14 & foreach code \\
\hline
15 & goto code not executed \\
\hline
16 & goto code executed \\
\hline
17 & function body executes code \\
\hline
18 & input passed via function then code \\
\hline
19 & code then output passed via function \\
\hline
20 & class body executes code \\
\hline
21 & input passed via class then code \\
\hline
22 & code then output passed via class \\
\hline
\end{tabular}
\caption{IDs and Code Description of Complexities Defined for PHP}
\label{tab:complexity IDs for PHP}
\end{table}


Here are the conditions currently available to be used in code complexities.
We explain condition modules in Sec.~\ref{sec: condition modules}.
Table~\ref{tab:condition IDs for PHP} shows the ID, the code, and whether it
always evaluates to true or false.

\begin{table}[H]
\centering
\begin{tabular}{|r|l|l|}
\hline
\textbf{ID} & \textbf{Code} & \textbf{Value} \\
\hline
1 & \verb|1==1| & True \\
\hline
2 & \verb|1==0| & False \\
\hline
\end{tabular}
\caption{IDs, Code, and Value to Which it Evaluates of Conditions Defined for
  PHP}
\label{tab:condition IDs for PHP}
\end{table}


\section{Python language}
\label{sec:Python language}

This documents the flaws, flaw groups, conditions, and complexities currently in the
Python language files.

{\Large FILL IN}

% make this begin on a new page (not a new column) even in
% two-column mode
\clearpage

\section{Contents of git Repository}
\label{gitContent}

\begin{figure}[htbp]
  \includegraphics[width=1\linewidth]{fig_git_files.png}
  \caption{Snapshot of files in the VTSG git repository, which is at
    \href{https://github.com/usnistgov/VTSG}{https://github.com/ usnistgov/VTSG},
    as of 8 March 2022.}
  \label{fig:git files}
\end{figure}

\begin{figure}[htbp]
  \includegraphics[width=0.8\linewidth]{fig_README_md.png}
  \caption{README.md file, which is at
    \href{https://github.com/usnistgov/VTSG/blob/master/README.md}
         {https://github.com/usnistgov/VTSG/blob/master/ README.md},
    as of 8 March 2022.}
  \label{fig:README.md file}
\end{figure}


%\section*{Appendix B: Change Log}
%\addcontentsline{toc}{section}{Appendix B: Change Log}
%If updating document with errata, detail changes made to document – delete if not applicable. \\

\end{document}
